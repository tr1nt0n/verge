\documentclass[10pt]{article}
\usepackage{fontspec}
\usepackage[utf8]{inputenc}
\setmainfont{Bodoni 72 Book}
\usepackage[paperwidth=9in,paperheight=12in,margin=1in,headheight=0.0in,footskip=0.5in,includehead,includefoot,portrait]{geometry}
\usepackage[absolute]{textpos}
\TPGrid[0.5in, 0.25in]{23}{24}
\parindent=0pt
\parskip=12pt
\usepackage{nopageno}
\usepackage{graphicx}
\graphicspath{ {./images/} }
\usepackage{amsmath}
\usepackage{tikz}
\newcommand*\circled[1]{\tikz[baseline=(char.base)]{
            \node[shape=circle,draw,inner sep=1pt] (char) {#1};}}

\begin{document}

\begingroup
\begin{center}
\huge NOTES FOR THE INTERPRETER
\end{center}
\endgroup

\begingroup
\begin{center}
\huge 
\end{center}
\endgroup


\begingroup
\textbf{General: 1.)} Dynamics in this score are effort dynamics, representing the physical force behind an action rather than the sounding dynamic. \textbf{2.)} Dashed arrows above the staff indicate a gradual transition from one technique to another. \textbf{3.)} Playing techniques persist until cancelled by another technique. \textbf{4.)} Dashed slurs indicate for the enclosed passage to be played legato, without prescribing a particular bowing. \textbf{5.)} A time signature of \textbf{X/X} indicates an ametric passage. The durations within these passages should still be counted in relation to the prescribed tempo, but no downbeats are indicated. Each of the notes in lowest staff of these passages counts as a ``measure'' for ease of identification within the passage. \textbf{6.)} Tuplet brackets which are open on the right side indicate the prolation of the note alone, rather than the number of beats within the prolation. For example, an open tuplet bracket reading ``3:2'' over a half note denotes a ``third" note which is one third of a whole note, or a triplet half note. \textbf{7.)} Grand pause fermate are counted as measures. \textbf{8.)} Flat glissandi are sometimes used for the same function as ties. \\
\endgroup 

\begingroup
\textbf{The Staves: 1.)} The interpreter may read up to three staves indicating separate voicings or rhythmic prolations. Simultaneities across these staves should be preserved in the form of double stops if possible, otherwise they should be arpeggiated, with emphasis placed on the rhythmically subsequent note. \textbf{2.)} If the notes in the upper staff are colored red, this indicates a parametric notation, wherein the upper staff indicates the actions of the right hand, and the lower staff indicates the actions of the left hand. \textbf{3.)} Dynamic markings in one staff apply to all active staves at a given moment.  \textbf{4.)} A four-line staff with an unpitched clef indicates playing on the open strings, wherein the top line indicates string I, the next line indicates string II, and so on. \\
\endgroup

\begingroup
\textbf{Microtones: 1.)} An inverted flat accidental indicates to play a quarter tone flat, and a sharp accidental with only one vertical line indicates to play a quarter tone sharp. \textbf{2.)} Justly tuned passages are notated using Helmholtz-Ellis accidentals with cent deviations from an equally tempered note printed above. In the absence of an electric tuner, approximations of these deviations are acceptable. \\
\endgroup

\begingroup
\textbf{Note Heads: 1.) Finger pressure of the left hand} is indicated by note head shape, wherein round note heads indicate a fully closed string, triangular note heads indicate a finger pressure between harmonic pressure and fully closing the string, and diamond-shaped note heads indicate to touch the string on the pitch indicated with finger pressure as if playing a harmonic, whether a harmonic sounds or not. \textbf{2.)} Cross-shaped note heads indicate to damp the string near the pitch indicated with the left hand, removing as much pitch from the sound as possible. \\
\endgroup

\begingroup
\textbf{Bowing Indications and Abbreviations: 1.) Dietro ponticello}, or \textbf{DP} indicates to bow between the bridge and the tailpiece. This indication is always coupled with a direction to bow on the wrapping, creating a screech-like sound similar to a shrill scratch tone. \textbf{2.) Ordinario}, or \textbf{Ord.} cancels dietro ponticello. \textbf{3.) Col legno tratto}, or \textbf{CLT} indicates to rub with the wood of the bow, and \textbf{1/2 col legno tratto} or \textbf{1/2 CLT} indicates to rub with both the wood and the hair. These directions are always coupled with a circle bow articulation, which indicates to draw the bow in a full circle on the string, completing the rotation within the duration of the articulated note. \textbf{4.) Crine} indicates to rub with the hair of the bow only. \textbf{5.) Extra slow bow}, or \textbf{XSB} indicates to bow as slowly as possible, generating scratch tone at higher bow pressures. This is coupled with an indication reading ``approx. 3 clicks per s." at measure 86, which indicates to draw the bow so slowly and at such high pressure that only semi-periodic clicks of bow friction appear. In this passage, the bow is drawn faster and faster until the pitch of the note is uncovered from the noise at measure 88. \textbf{6.) Normale bow}, or \textbf{NB} indicates normale bowing speed.  \\
\endgroup

\end{document}